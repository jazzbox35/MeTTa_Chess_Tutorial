\documentclass{article}

% ----------- core mathematics -----------
\usepackage{amsmath}   % align, split, cases, etc.
\usepackage{amssymb}   % \mathbb, \mathcal, \leqslant, \geqslant ...
\usepackage{amsfonts}  % blackboard bold, fraktur if desired
\usepackage{graphicx}  % for future figures, even if none are included yet
\usepackage{listings}  % for code examples
\usepackage{hyperref}  % clickable cross-refs; load last
\usepackage{mdframed}  % for boxed content

% ----------- tables & arrays ------------
\usepackage{array}     % extended column specifiers in tabular
\usepackage{booktabs}  % nicer horizontal rules (optional; you may keep \hline)
\usepackage{multirow}  % multi-row cells if you extend the tables later

% ----------- layout & floats ------------
\usepackage{geometry}  % easy margin control (defaults are fine; optional)
\usepackage{caption}   % better caption spacing for tables/figures

\title{You Only Need These Few Constructs}
\order{4}

\begin{document}

\section{You Only Need These Few Constructs}

First some good news! Solid and potent MeTTa code can be built with only about a dozen fairly simple programming constructs. The possible downside though is that MeTTa, while a unique language, does look somewhat like LISP with nested, balanced parenthesis to mind. However, with a bit of practice and an IDE that tracks parenthesis nesting with colors you'll do fine.

In this section we will examine code constructs to add and manipulate atoms, assign variables, define rewrite rules, handle conditional execution, run a sequence of statements, loop, and manage the combinatorial explosion with the superpose, match, and collapse functions. 

We'll use these construct types to create the chess game in the next section.

\subsection{Add and Manipulate Atoms}

First, the miner computes the total number of ground atoms:

\begin{verbatim}
;; Count every atom in the database
(= (universe-size)
   (let $u (collapse (match $dbspace $x 1))
       (tuple-count $u)))
\end{verbatim}








\end{document}
