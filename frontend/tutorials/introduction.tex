\documentclass{article}

% ----------- core mathematics -----------
\usepackage{amsmath}   % align, split, cases, etc.
\usepackage{amssymb}   % \mathbb, \mathcal, \leqslant, \geqslant ...
\usepackage{amsfonts}  % blackboard bold, fraktur if desired
\usepackage{graphicx}  % for future figures, even if none are included yet
\usepackage{listings}  % for code examples
\usepackage{hyperref}  % clickable cross-refs; load last
\usepackage{mdframed}  % for boxed content
\usepackage{xcolor}    % for colored text/links

% ----------- tables & arrays ------------
\usepackage{array}     % extended column specifiers in tabular
\usepackage{booktabs}  % nicer horizontal rules (optional; you may keep \hline)
\usepackage{multirow}  % multi-row cells if you extend the tables later

% ----------- layout & floats ------------
\usepackage{geometry}  % easy margin control (defaults are fine; optional)
\usepackage{caption}   % better caption spacing for tables/figures

\title{What is MeTTa and why use it?}
\order{1}
\author{Mike Archbold}
\date{December 2025}

\begin{document}


\section{What is MeTTa and why use it?}

\subsection{It Might Not Be Clear What MeTTa Is}

Perhaps you tried a bit of MeTTa coding, or heard about MeTTa on social media, at a conference, or from a colleague, and want to explore. You heard it is an innovative AI language that would benefit your projects. 

You probably have some questions in mind such as:

\begin{itemize}
  \item What is the use of a symbolic AI language in an LLM age? Why learn it?
  \item It looks a bit like LISP. Is MeTTa a LISP variant?
  \item How do I install MeTTa on my machine?
  \item How can I understand the code? How many constructs do I really need for coding?
  \item What does "MeTTa" mean?
  \item Somebody said it was too slow. Is that still true? (No)
\end{itemize}

{\itshape Beginning users might want to visit \href{https://metta-lang.dev/docs/learn/learn.html}{\textcolor{blue}{MeTTa Learn Docs}} for additional MeTTa background.}

Well, if you have variations of these questions, you are on the right tutorial!


\subsection{It's Like a Fusion of SQL and Ordinary Code}

The single most defining characteristic of MeTTa is this: it insists on giving you all possible results. Whether you like it or not!

Suppose you are creating a package delivery routing application. Given a starting point and a destination, let's say ten routes exist, each with many twists and turns, so your code has lots of constraints. MeTTa outputs all ten complete routes in one run, at once. 

Now suppose you are creating a scheduling application. Given today's agenda, there are a ninety-nine possible schedules given available rooms, equipment, personnel, tasks, etc. Again, MeTTa outputs all ninety-nine schedules in a single result.

If this sounds to you a bit like an SQL "return all rows" matching request, you'd be on the right track. In essence, MeTTa is a fusion of "return all rows" SQL type calls and basically ordinary code. Given some specific inputs, code written in MeTTa by default will return all possible output combinations that result from executing all possible paths in your code applicable to the input. You get everything.

This might immediately raise the worrying prospect of having to deal with an avalanche of results. Well, don't worry, because there are ways of corralling such an avalanche with MeTTa. In fact that's the point of it. And you basically only need about a dozen constructs to do everything you need in MeTTa.

\subsection{Corralling the Combinatorial Explosion}

Reality is incredibly complex. AI is a distinct discipline within computer science because worldly complex problems can't be handled by conventional programming in which all possible combinations are accounted for and pre-coded in advance. Techniques are developed to reach goals without requiring the specific pre-coding and foresight of all sequences of all possible combinations. AI evolved into a trick bag for handling such situations. AI, symbolic and neural, has to search for a path from start request to goal result without a predetermined route. 

Historically it has been difficult to write conventional symbolic programs to handle problems characterized by real world complexity. The strategy MeTTa employs to meet the challenge of subduing combinatorial explosions is to add a small set of programming principles and constructs to enable conventional seeming programs to tame a massive number of combinations and "all possible results." (You can get similar results with languages such as Prolog, but Prolog is a very unusual language for beginning and potential users to learn.)

\subsection{Superpose Match Collapse}

There are three defining programming constructs in MeTTa to manage combinatorial explosion, covered in detail later along with a few other supporting constructs:

\begin{enumerate}
  \item \textbf{superpose}. Specify a set of cases to evaluate,
  \item \textbf{match}. Determine all results for specified cases,
  \item \textbf{collapse}. Return all results for filtering/processing.
\end{enumerate}

Basically the combinatorial explosion happens as a result of the first two steps, but \textbf{collapse} sets an important boundary. If you forget to terminate this sequence with \textbf{collapse,} beware that MeTTa might insist on backing up later in your code and recomputing other variations you may not have thought of, much to your surprise and possible annoyance. 

In order to avoid this potential "gotcha" (MeTTa unexpectedly backing up, retrying your match) all you need to do is use \textbf{collapse}. MeTTa will then proceed forward and not dwell on other possible cases. \textbf{collapse} is your primary tool for managing the combinatorial explosion, and can be used basically anywhere in MeTTa code in which multiple results are possible.

BENFITS OF METTA, DISADVANTAGES

LLM ADVANTAGES/DIS-
and HOW THEY CAN REPOSE TOGETHER


\section{How to Use the Tutorial}

\subsection{Prerequisites}

Some background exposure to MeTTa, LISP, Prolog or similar languages is helpful. Absolute newbies might want to visit \href{https://metta-lang.dev/docs/learn/learn.html}{MeTTa Learn Docs} for additional MeTTa background, however the tutorial covers most basics. Knowing some chess too, is helpful, but you can still gain knowledge of MeTTa techniques even without knowing how to play.

\subsection{Chess}

Chess has a high degree of combinatorial explosion. After a crash course in a few basic MeTTa constructs, we will code a chess game. After we have created a basic game, you can then make a major modification to it. You can click the button above to play chess using your developed game, or reset to the default program code that is supplied with the tutorial and play using that code.

\subsection{AtomSpace}

Your program and data (eg., chessboard) reside together in atomspace. For this tutorial you can use the handy boxes appearing throughout to try out segments of code. You can display and reset your atomspace using the buttons on the far upper right of the tutorial at any time.

In MeTTa, the "code" is essentially rewrite rules and is stored in atomspace. Rewrite rules start with "(=". Given some input pattern, it executes. The input pattern below is just (hello-world).

Your "data," is stored in atomspace enclosed in balanced parenthesis. 

If you start your MeTTa input with "!(..." you trigger an immediate execution. 

In the example below, the "!("... construct calls the "(=" rewrite rule. When you run the example, (Hello World) is returned. 

When you click the "Run" button below, given the example, MeTTa will add the "code" and "data" statements to atomspace. Then MeTTa executes !(hello-world).

Importantly, note that both code and data are considered atoms and reside in atomspace together for manipulation.

A red parenthesis is a tip-off that you have unbalanced parens somewhere in your expression.

Note: it's possible to use different spaces inside atomspace, but for this tutorial we will only use &self.

\begin{verbatim}
; "code" rewrite rule atom
(= (hello-world) (Hello World))  

; "data," another type of atom
(MeTTa is fun!)

; execute
!(hello-world)
\end{verbatim}

\section{Tutorial Objectives}


\begin{enumerate}
  \item \textbf{Why} Understand the potential of MeTTa in an age of LLMs. 
  \item \textbf{What} Gain a basic understanding of STRUCTURE MEANS
  \item \textbf{How} CODING ABLE TO SET UP SERVER
\end{enumerate}


\end{document}
