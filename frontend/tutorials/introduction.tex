\documentclass{article}

% ----------- core mathematics -----------
\usepackage{amsmath}   % align, split, cases, etc.
\usepackage{amssymb}   % \mathbb, \mathcal, \leqslant, \geqslant ...
\usepackage{amsfonts}  % blackboard bold, fraktur if desired
\usepackage{graphicx}  % for future figures, even if none are included yet
\usepackage{listings}  % for code examples
\usepackage{hyperref}  % clickable cross-refs; load last
\usepackage{mdframed}  % for boxed content

% ----------- tables & arrays ------------
\usepackage{array}     % extended column specifiers in tabular
\usepackage{booktabs}  % nicer horizontal rules (optional; you may keep \hline)
\usepackage{multirow}  % multi-row cells if you extend the tables later

% ----------- layout & floats ------------
\usepackage{geometry}  % easy margin control (defaults are fine; optional)
\usepackage{caption}   % better caption spacing for tables/figures

\title{Introduction }
\order{1}
\author{Tutorial Team}
\date{May 21, 2025}

\begin{document}


\section{Introduction}

1) Introduction. Immediately make a clear case for why a developer would want to use MeTTa. We will
highlight its advantages: accuracy, ability to create/read/update/delete in a space of high combinatorial
explosion not subject to hallucinations. It can ultimately be used with an MCP interface as part of a
broader system design. Then describe a minimum of prerequisites (referring to other tutorials such as
the first couple on the Metta-Learn webpages). However, we will try to make the tutorial as
prerequisite-free as possible. Some Lisp or other AI language experience is helpful but should not
strictly be necessary.


\section{Implementation Details}

This whole process is implemented in MeTTa, so if one runs it within a fast MeTTa interpretation or compilation framework, it should scale to huge numbers  of Atoms.  (Different Atoms may have different sizes, but very roughly, in a server with terabytes of RAM one may be able to host Atomspaces with 100B or more Atoms.)   It also exposes a clean API so you can integrate pattern mining directly into larger learning or reasoning pipelines.

\subsection{Current Capabilities and Future Work}

In its current form the Pattern Miner runs against an Atomspace in RAM on a single machine, however the underlying algorithms are well designed for extension to a distributed-processing setting, and we anticipate future Pattern Miner versions working effectlvely on distributed Atomspaces as well.   This further development will be done in conjunction with other existing tools like the Distributed Atomspace (DAS) XX and the Mettacycle decentralized infrastructure (XX).

\subsection{The Atomspace You Will Be Working On}

All of the live examples on this tutorial operate within a single AtomSpace. As you run each code example, or "cell," you are continuously adding new information to this space, building upon the results from all the cells you have already run. The "Reset Atomspace" button wipes the whole atomspace and lets you start on a clean slate whilst the "reset" button on the cells (code snippets) themselves  provide a way to start over from that specific point whilst maintaing the state of the atom upto that particular snippet. The most common use case for these buttons is if you happen to be facing an import issue at some point you could reset the whole atomspace or reset the snippet that executes the importing tasks so that subsequent cells can get the correct imports they need. 

\section{Tutorial Objectives}

In this tutorial you will learn to:

\begin{enumerate}
  \item \textbf{Ground yourself} in the core concepts (AtomSpace, patterns, frequency, and surprisingness).
  \item \textbf{Step through} the full mining pipeline--from abstracting link-nodes to scoring surprisingness--using diagrams and MeTTa code.
  \item \textbf{Run hands-on} MeTTa scripts to install, configure, and execute the Hyperon Pattern Miner on your own AtomSpace data.
  \item \textbf{Explore advanced use cases} and performance-tuning strategies so you can apply pattern mining to real-world datasets.
\end{enumerate}

By the end of this tutorial, you'll understand what the Hyperon Pattern Miner is, why it matters, and how it works--and be at least a little bit experienced with some hands-on examples.   You should then be prepared to work on some larger examples on your own if that's the direction your work or interest leads you!

\end{document}