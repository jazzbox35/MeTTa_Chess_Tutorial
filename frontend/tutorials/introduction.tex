\documentclass{article}

% ----------- core mathematics -----------
\usepackage{amsmath}   % align, split, cases, etc.
\usepackage{amssymb}   % \mathbb, \mathcal, \leqslant, \geqslant ...
\usepackage{amsfonts}  % blackboard bold, fraktur if desired
\usepackage{graphicx}  % for future figures, even if none are included yet
\usepackage{listings}  % for code examples
\usepackage{hyperref}  % clickable cross-refs; load last
\usepackage{mdframed}  % for boxed content

% ----------- tables & arrays ------------
\usepackage{array}     % extended column specifiers in tabular
\usepackage{booktabs}  % nicer horizontal rules (optional; you may keep \hline)
\usepackage{multirow}  % multi-row cells if you extend the tables later

% ----------- layout & floats ------------
\usepackage{geometry}  % easy margin control (defaults are fine; optional)
\usepackage{caption}   % better caption spacing for tables/figures

\title{What is MeTTa and why use it?}
\order{1}
\author{Mike Archbold}
\date{December 2025}

\begin{document}


\section{What is MeTTa and why use it?}

\subsection{It Might Not Be Clear What MeTTa Is}

Perhaps you heard about MeTTa on social media, at a conference, or from a colleague, and want to explore. You heard it is an innovative AI language that would benefit your projects. 

You probably have some questions in mind such as:

\begin{itemize}
  \item What is the use of a symbolic AI language in an LLM age? Why learn it?
  \item It looks a bit like LISP. Is MeTTa a LISP variant?
  \item How do I install MeTTa on my machine?
  \item How can I understand the code? How many constructs do I really need for coding?
  \item What does "MeTTa" mean?
  \item Somebody said it was too slow. Is that still true? (No)
\end{itemize}

Well, if you have variations of these questions, you are on the right tutorial!


\subsection{It's Like a Fusion of SQL and Ordinary Code}

The single most defining characteristic of MeTTa is this: it insists on giving you all possible results. Whether you like it or not!

Suppose you are creating a package delivery routing application. Given a starting point and a destination, let's say ten routes exist, each with many twists and turns, so your code has lots of constraints. MeTTa outputs all ten complete routes in one run. 

Now suppose you are creating a scheduling application. Given today's agenda, there are a ninety-nine possible schedules given available rooms, equipment, personnel, tasks, etc. Again, MeTTa outputs all ninety-nine schedules in one run.

If this sounds to you a bit like an SQL "return all rows" type of call, you'd be on the right track. In essence, MeTTa is a fusion of "return all rows" SQL type calls and ordinary code. Given some specific inputs, code written in MeTTa by default will return all possible output combinations using all possible paths through your code.

This might immediately raise the worrying prospect of having to deal with an avalanche of results. Well, don't worry, because there are ways of corralling MeTTa. Basically you only need about a dozen constructs to do everything you need in MeTTa.

But even if there are ways of controlling the output, what is the use? Well, the reason AI is a distinct discipline within computer science is because worldly problems can't be handled by typical procedural code in which all possible input combinations and desired outputs are accounted for in advance. Reality is an incredibly complex phenomena. There are just too many possible combinations. We can only pre-code in advance for problems that are highly constrained. So, that is why AI evolved into a trick bag of techniques designed to handle such combinatorial explosions.
\subsubsection{AtomSpace}

The \textbf{AtomSpace} is Hyperon's persistent metagraph store.  Every node and link you mine lives there.  In MeTTa you typically create or bind an AtomSpace via:

\begin{verbatim}
!(import! &self hyperon-miner:match:MinerMatch)
!(bind! &atom-space (new-space)) 
\end{verbatim}



\section{Implementation Details}

This whole process is implemented in MeTTa, so if one runs it within a fast MeTTa interpretation or compilation framework, it should scale to huge numbers  of Atoms.  (Different Atoms may have different sizes, but very roughly, in a server with terabytes of RAM one may be able to host Atomspaces with 100B or more Atoms.)   It also exposes a clean API so you can integrate pattern mining directly into larger learning or reasoning pipelines.

\subsection{Current Capabilities and Future Work}

In its current form the Pattern Miner runs against an Atomspace in RAM on a single machine, however the underlying algorithms are well designed for extension to a distributed-processing setting, and we anticipate future Pattern Miner versions working effectlvely on distributed Atomspaces as well.   This further development will be done in conjunction with other existing tools like the Distributed Atomspace (DAS) XX and the Mettacycle decentralized infrastructure (XX).

\subsection{The Atomspace You Will Be Working On}

All of the live examples on this tutorial operate within a single AtomSpace. As you run each code example, or "cell," you are continuously adding new information to this space, building upon the results from all the cells you have already run. The "Reset Atomspace" button wipes the whole atomspace and lets you start on a clean slate whilst the "reset" button on the cells (code snippets) themselves  provide a way to start over from that specific point whilst maintaing the state of the atom upto that particular snippet. The most common use case for these buttons is if you happen to be facing an import issue at some point you could reset the whole atomspace or reset the snippet that executes the importing tasks so that subsequent cells can get the correct imports they need. 

\section{Tutorial Objectives}

In this tutorial you will learn to:

\begin{enumerate}
  \item \textbf{Ground yourself} in the core concepts (AtomSpace, patterns, frequency, and surprisingness).
  \item \textbf{Step through} the full mining pipeline--from abstracting link-nodes to scoring surprisingness--using diagrams and MeTTa code.
  \item \textbf{Run hands-on} MeTTa scripts to install, configure, and execute the Hyperon Pattern Miner on your own AtomSpace data.
  \item \textbf{Explore advanced use cases} and performance-tuning strategies so you can apply pattern mining to real-world datasets.
\end{enumerate}

By the end of this tutorial, you'll understand what the Hyperon Pattern Miner is, why it matters, and how it works--and be at least a little bit experienced with some hands-on examples.   You should then be prepared to work on some larger examples on your own if that's the direction your work or interest leads you!

\end{document}
