\documentclass{article}
\usepackage{amsmath}
\usepackage{amssymb}
\usepackage{graphicx}
\usepackage{listings}
\usepackage{hyperref}

\title{MeTTa Components for a Simple Chess Game}
\order{3}

\begin{document}


\section{How to Create a Simple Chess Game}

Now armed with a few MeTTa constructs, we'll jump into practical MeTTa software development. 

We will be using a backend MeTTa server running in the cloud. Gameplay is handled through your browser which sends only two commands to the backend MeTTa server: M and S which handle piece (M)oving and (S)tart/reset respectively. These commands are defined in the "Command handler functions for gameplay" section below.

Note that we won't be dynamically redefining functions, although it's possible in MeTTa. To create our chess program, we need to define the following:

\begin{itemize}
  \item \texttt{Utility functions}
  \item \texttt{Constants for use throughout the program's run}
  \item \texttt{Chess functions}
  \item \texttt{Command handler functions for gameplay}
\end{itemize}

The chess program is stored in atomspace using your browser. 

If at any time you wish to reset atomspace to the default MeTTa chess program just click "Reset Atomspace" and then click "Play Chess". If "Reset Atomspace" is greyed-out, there is no need to click.

In general, if you see unexpected results or an error, try clicking "Reset Atomspace" and proceeding with examples. 

Please click "Reset Atomspace"  (if enabled) now to start with a clean slate.

\subsection{Utility functions}

First we need to define a few general functions to be used by the whole program.

\begin{verbatim}
    
;***************************************************************
; Function:     nth
; Description:  Returns the N-th element from a list (1-based index).
;
; Input:        $n     - The position (1-based) of the element to return.
;               $list  - A list of elements (e.g., (a b c d)).
;
; Output:       The atom at position $n in the list. Returns the first element if n = 1.
;***************************************************************
(= (nth $n $list) 
    (if (== $n 1)
        (car-atom $list)
        (nth (- $n 1) (cdr-atom $list)))) ; Recursion: move to the next element (cdr-atom) and decrease n.

;***************************************************************
; Function:     contains_symbol
; Description:  Checks whether a given symbol exists in a list.
;
; Input:        $list - A list of atoms (e.g., (a b c d))
;               $sym  - A symbol to search for in the list
;
; Output:       True if $sym is found in $list; otherwise False
;***************************************************************
(= (contains_symbol $list $sym) 
    (if (== $list ())
        False
        (if (== (car-atom $list) $sym)
            True
            (contains_symbol (cdr-atom $list) $sym))))

\end{verbatim}


\subsection{Putting It All Together}



\begin{verbatim}
(end)
\end{verbatim}

Summary


\end{document}
