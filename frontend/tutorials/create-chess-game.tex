\documentclass{article}
\usepackage{amsmath}
\usepackage{amssymb}
\usepackage{graphicx}
\usepackage{listings}
\usepackage{hyperref}

\title{MeTTa Components for a Simple Chess Game}
\order{3}

\begin{document}


\section{How to Create a Simple Chess Game}

Now armed with a few MeTTa constructs, we'll examine practical MeTTa software development. 

\subsection{Reset Atomspace to Default Greedy Chess Program}

For this tutorial section we will be using the default MeTTa greedy chess program. To use the default chess program \textbf{please click "Reset Atomspace" if enabled (not greyed-out) and then click "Play Chess".} If "Reset Atomspace" is greyed-out, just click "Play Chess". If you are taken to the chessboard tab, return to this tab. 

In general, if you see unexpected results, simply clicking "Reset Atomspace" and then "Play Chess" will reset atomspace to the default program.

\subsection{Backend MeTTa server}

We will be using a backend MeTTa server running in the cloud. Gameplay is handled through your browser which sends only two commands to the backend MeTTa server: M and S which handle piece (M)oving and (S)tart/reset respectively. These commands are defined in the "Command handler functions for gameplay" section below.

\subsection{Program Components}

Note that we won't be dynamically redefining functions, although it's possible in MeTTa. For our chess program, the following categories are defined:

\begin{itemize}
  \item \texttt{Utility functions}
  \item \texttt{Constants for use throughout the program's run}
  \item \texttt{Chess related functions}
  \item \texttt{Command handler functions for gameplay}
\end{itemize}

The chess program is stored in atomspace using your browser (in a format maintained by the MeTTa server). We will be covering only the most salient functions in the chess program herein. However, in addition to displaying atomspace by clicking the button above, you can download the full source code by clicking on the "Your Turn to Improve the Game" section and look for the download link. 

\subsubsection{Utility Functions}

First we need to define a few general functions to be used by the whole program.

Our first couple of functions are standard LISP-ish looking tail recursion functions. In MeTTa \textbf{car-atom} and \textbf{cdr-atom} (as well as \textbf{cons-atom}) are effectively the same as LISP but with -"atom" appended. If you need a refresher or crash course in these constructs an LLM would help. \textbf{car-atom} gives you the first symbol in a list while \textbf{cdr-atom} the rest of the list, still in parenthesis. The main trick is that you code the end boundary case and the general recursive
case for items remaining in the list. Fortunately, you do less tail recursion searching in MeTTa than LISP but
it is still a handy way of looping through a list.

The "nth" function, below, is used constantly in the program to extract piece details.

The chess program uses the following format universally for board squares:

\textbf{(x y color piece)} (occupied square)

\textbf{(x y)} (unoccupied square)

If length is 4, the square is occupied. If length is 2, unoccupied. x and y are standard Cartesian coordinates.
Piece color values are either s for silver (human) or g for gold (AI). Piece types are r, n, b, q, k, p for 
rook, knight, bishop, queen, king, and pawn respectively.


\begin{verbatim}
; HIDE-RUN    
;***************************************************************
; Function:     nth
; Description:  Returns the N-th element from a list (1-based index).
;
; Input:        $n     - The position (1-based) of the element to return.
;               $list  - A list of elements (e.g., (a b c d)).
;
; Output:       The atom at position $n in the list. Returns the first element if n = 1.
;***************************************************************
(= (nth $n $list) 
    (if (== $n 1)
        (car-atom $list)
        (nth (- $n 1) (cdr-atom $list)))) ; Recursion: move to the next element (cdr-atom) and decrease n.
\end{verbatim}

\begin{verbatim}
; reset chess game to initial state
!(S)

; find x axis value for this silver pawn located in square (3, 2)
!(nth 1 (3 2 s p))   

; find the piece type
!(nth 4 (3 2 s p))   
\end{verbatim}

Below is a fairly ordinary recursive function used in the chess program which checks for a given symbol in a provided list.

\begin{verbatim}
; HIDE-RUN
;***************************************************************
; Function:     contains_symbol
; Description:  Checks whether a given symbol exists in a list.
;
; Input:        $list - A list of atoms (e.g., (a b c d))
;               $sym  - A symbol to search for in the list
;
; Output:       True if $sym is found in $list; otherwise False
;***************************************************************
(= (contains_symbol $list $sym) 
    (if (== $list ())
        False
        (if (== (car-atom $list) $sym)
            True
            (contains_symbol (cdr-atom $list) $sym))))

\end{verbatim}



\begin{verbatim}
!(S)

; see if there is a gold piece in board square (1 8) 
!(contains_symbol (1 8) g)

\end{verbatim}

Our next function \textbf{int_to_char} is a simple case statement to determine the x-axis letter that 
corresponds to the numeric Cartesian x value. The program returns conventional letters for the chess x-axis.

\begin{verbatim}
; HIDE-RUN    
;***************************************************************
; Function:     int_to_char
; Description:  Converts an integer 1..8 to the corresponding incremental letter.
;
; Input:        $int - Integer in range 1 to 8
;
; Output:       Corresponding file letter "a".."h", or "?" if invalid
;***************************************************************
(= (int_to_char $int)
    (case $int
        (
        (1 A)
        (2 B)
        (3 C)
        (4 D)
        (5 E)
        (6 F)
        (7 G)
        (8 H)
        ($_ ?) ; fallback for invalid input
        )))

\end{verbatim}

\begin{verbatim}
; initialize program
!(S)

;example int_to_char call        
!(int_to_char 5)
\end{verbatim}

\subsubsection{Declaring Constants and Initialization for the Run}

An easy way to define constants is just by coding them. These can be looked up in the \textbf{&self} 
space using \textbf{match}.

\begin{verbatim}
;HIDE-RUN
;*******************************************************
; Constants 
;*******************************************************

(highestrank k) 
(highrank q) 
(medrank r) 
(medrank b) 
(medrank n) 
(lowrank p) 

(rank k) 
(rank q) 
(rank r) 
(rank b) 
(rank n) 
(rank p) 

\end{verbatim}

\begin{verbatim}
!(S) ;reset for demo

;example to return all ranks in a list. You get all ranks because MeTTa insists on giving all the results!
!(collapse (match &self (rank $rank) $rank))
\end{verbatim}

The following expression adds an atom for game state. Technically it's not a constant since we will be changing it later
by removing and adding an atom of the form "(game-state -value-)". So you can basically slip \textbf{de facto global variables} into your program just by declaration with the first symbol acting as the variable name and the second its value. In order to change its value you just remove the entire two symbol atom list and add back to atomspace with a different second symbol value. This is shown further below.

\begin{verbatim}
; HIDE-RUN
;*******************************************************
; Declare initializing game state
; (game-state <value>) is a de facto global variable
;*******************************************************

(game-state initializing)

\end{verbatim}


\subsubsection{Chess Related Functions}

The program persists two major data structures between moves. One is the chessboard state, the second game state:

1) \textbf{(board-state (  (1 1 s r) (1 2 s n).... (4 4).... (1 8 g r)... ))} with 64 sublists

2) \textbf{(game-state -value-)}

The "board-state" list is not used as the data structure for searching. Instead, when the AI is experimenting with possible next moves, small standalone atoms in atomspace are created for each square of this form:

\textbf{(square x y color rank) and (square x y)} to denote both occupied and unoccupied squares.

The reason the longer "board-state" list of 64 elements representing all squares on the chessboard is \textbf{not} used for deep searches is because \textbf{METTA IS FASTER LOOKING UP INDIVIDUAL ATOMS THAN CHUGGING THROUGH LISTS.} A defining feature of MeTTa is random access in atomspace, not LISP-ish tail recursion search.

The following functions are used for the input of the 64-element board list and output of temporary 
corresponding atomic "square" (as above) atoms for all chess pieces. The deep searching in the program uses these 
atomic "square" atoms for random access speed. 

This example, the "add pieces" function, converts each square of our full 8 x 8 chessboard state list into individual atoms for quick lookup. Here we use the \textbf{let*} construct. Note that \textbf{cons-atom} inserts "square" into the individual, atomic square list. The third variable is a dummy variable, unused.

\begin{verbatim}
; HIDE-RUN
;***************************************************************
; Function:     add-pieces
; Description:  Converts a board representation into atomspace facts.
;               Each square from the board list becomes a `(square ...)` atom.
;
; Input:        $board - A list of square definitions:
;                        Each square is either (x y) for empty or (x y color piece) if occupied.
;
; Output:       Adds each square to the atomspace as a `(square ...)` atom.
;               Returns True after processing the entire board.
;***************************************************************
(= (add-pieces $board)
  (if (== $board ())
      True
      (let* (
            ($next-square (car-atom $board))
            ($next-square-atom (cons-atom square $next-square))
            ($_ (add-atom &self $next-square-atom))
            )
      (add-pieces (cdr-atom $board))))) ;recurse here to create a loop to process all squares 

\end{verbatim}


\begin{verbatim}
!(S) ;reset for demo

; Here we call add-pieces to create individual square atoms for each piece. Display atomspace after running this command and you should see individual atoms of the form (square x y ...)
!(add-pieces ((1 8 g r) (2 8 g n) (3 8 g b) (4 8 g q) (5 8 g k) (6 8 g b) (7 8 g n) (8 8 g r) 
                        (1 7 g p) (2 7 g p) (3 7 g p) (4 7 g p) (5 7 g p) (6 7 g p) (7 7 g p) (8 7 g p) 
                        (1 6) (2 6) (3 6) (4 6) (5 6) (6 6) (7 6) (8 6) 
                        (1 5) (2 5) (3 5) (4 5) (5 5) (6 5) (7 5) (8 5) 
                        (1 4) (2 4) (3 4) (4 4) (5 4) (6 4) (7 4) (8 4) 
                        (1 3) (2 3) (3 3) (4 3) (5 3) (6 3) (7 3) (8 3) 
                        (1 2 s p) (2 2 s p) (3 2 s p) (4 2 s p) (5 2 s p) (6 2 s p) (7 2 s p) (8 2 s p) 
                        (1 1 s r) (2 1 s n) (3 1 s b) (4 1 s q) (5 1 s k) (6 1 s b) (7 1 s n) (8 1 s r))   
  )

; display occupied squares
!(collapse (match &self (square $x $y $color $piece) ($x $y $color $piece)))

; display unoccupied squares
!(collapse (match &self (square $x $y) ($x $y)))
\end{verbatim}

The "reset-pieces" function removes all the small "square" atoms and refreshes atomspace with a new set of "square" atoms using "add-pieces" and an input-supplied chessboard state.


\begin{verbatim}
; HIDE-RUN

;***************************************************************
; Function:     reset-pieces
; Description:  Clears existing board state from atomspace and
;               re-injects the current board list as square atoms.
;
; Input:        $board - The current board, as a list of square terms:
;                        Each term is (x y) for empty, or (x y color rank) for occupied.
;
; Output:       Atomspace is updated with the board's square atoms.
;***************************************************************
(= (reset-pieces $board)
    (progn
      ; remove prior occupied square atoms from atomspace
      (remove-atom &self (square $x $y $s $p))  
      ; remove empty squares too
      (remove-atom &self (square $x $y))  
      ; now add the present board's squares to atomspace.
      (add-pieces $board)))

\end{verbatim}


\begin{verbatim}
; refresh the 'square' atoms using the input chessboard below. 
; NOTE THERE HAS BEEN A MOVE WITH A PAWN IN SQUARE (4 4) 

!(collapse (reset-pieces
            ((1 8 g r) (2 8 g n) (3 8 g b) (4 8 g q) (5 8 g k) (6 8 g b) (7 8 g n) (8 8 g r) 
                        (1 7 g p) (2 7 g p) (3 7 g p) (4 7 g p) (5 7 g p) (6 7 g p) (7 7 g p) (8 7 g p) 
                        (1 6) (2 6) (3 6) (4 6) (5 6) (6 6) (7 6) (8 6) 
                        (1 5) (2 5) (3 5) (4 5) (5 5) (6 5) (7 5) (8 5) 
                        (1 4) (2 4) (3 4) (4 4 s p) (5 4) (6 4) (7 4) (8 4) 
                        (1 3) (2 3) (3 3) (4 3) (5 3) (6 3) (7 3) (8 3) 
                        (1 2 s p) (2 2 s p) (3 2 s p) (4 2) (5 2 s p) (6 2 s p) (7 2 s p) (8 2 s p) 
                        (1 1 s r) (2 1 s n) (3 1 s b) (4 1 s q) (5 1 s k) (6 1 s b) (7 1 s n) (8 1 s r))))
; display occupied squares
!(collapse (match &self (square $x $y $color $piece) ($x $y $color $piece)))

; display unoccupied squares
!(collapse (match &self (square $x $y) ($x $y)))

\end{verbatim}

The following two functions can be used to reset the value of our de facto global "game-state" variable.

\begin{verbatim}
; HIDE-RUN

;***************************************************************
; Function:     delete-prior-game-states
; Description:  Removes any existing `(game-state ...)` atoms from atomspace.
;               Used before setting a new game state.
;
; Input:        None
;
; Output:       Atomspace no longer contains any `game-state` atoms.
;***************************************************************
(= (delete-prior-game-states)
    (collapse (match &self (game-state $prior-state) (remove-atom &self (game-state $prior-state))))
)

;***************************************************************
; Function:     change-game-state
; Description:  Resets the current game state and sets a new one.
;
; Input:        $new-state - A symbol representing the new game state,
;                            e.g., `started`, `checkmate`, `restarted`, etc.
;
; Output:       Replaces any existing `game-state` atom with the new one.
;***************************************************************
(= (change-game-state $new-state)
  (progn
    (delete-prior-game-states) 
    (add-atom &self (game-state $new-state))))

\end{verbatim}

Note that "delete-prior-game-states" above uses a \textbf{collapse} to ensure ALL prior game state atoms are removed. Think of \textbf{collapse} as your means to force MeTTa to evaluate what is enclosed within its expression for all possible results. If MeTTa returns all possible results, it will be satisifed and not try to back up later and find some other variation that was missed.

After running this atomspace should show (game-state checkmate).

\begin{verbatim}

!(S) ;reset for demo

!(change-game-state checkmate)

\end{verbatim}

In the following function, "lots_of_pawns_in_home_row", there is a \textbf{match} construct used
inside the second argument of \textbf{let} to find all AI gold pawns (designated with a 'g') in row 7, the AI home row for pawns. Notice that our
friend the \textbf{collapse} encircles the match statement, as if saying "give me all the gold pawns in row 7 and don't return here for more execution later." The result of the second argument, a list of "square" atoms of the form (square x 7 g p) is assigned to the "pawns_at_home" variable. Then \textbf{size-atom} returns the length of the list and a boolean check returns true or false. Again, we don't have to loop through a list searching for particular squares -- we just use the formula of a \textbf{match} working on atomspace within an enclosing \textbf{collapse}. 


\begin{verbatim}
; HIDE-RUN
;***************************************************************
; Function:     lots_of_pawns_in_home_row
; Description:  Checks whether more than x AI (gold) pawns are still on row 7.
;               Used to influence early-game move strategy.
;
; Input:        None
;
; Output:       True if more than x (see below) pawns remain on row 7 (initial position);
;               otherwise False.
;***************************************************************
(= (lots_of_pawns_in_home_row)
   (let $pawns_at_home (collapse (match &self (square $x 7 g p) (square $x $y $s $p)))
        (if (> (size-atom $pawns_at_home) 6) ; <-- set here
          True
          False)))
\end{verbatim}

\begin{verbatim}

; We intialize the program for starting board state. Then a call to search the "square" atoms will reveal more than 6 in the AI's starting row of pawns. If you take the "g" and "p" out of a few of the pawns designated in row 7 the function will return false.

!(S) ;reset for demo

!(reset-pieces ((1 8 g r) (2 8 g n) (3 8 g b) (4 8 g q) (5 8 g k) (6 8 g b) (7 8 g n) (8 8 g r) 
                        (1 7 g p) (2 7 g p) (3 7 g p) (4 7 g p) (5 7 g p) (6 7 g p) (7 7 g p) (8 7 g p) 
                        (1 6) (2 6) (3 6) (4 6) (5 6) (6 6) (7 6) (8 6) 
                        (1 5) (2 5) (3 5) (4 5) (5 5) (6 5) (7 5) (8 5) 
                        (1 4) (2 4) (3 4) (4 4) (5 4) (6 4) (7 4) (8 4) 
                        (1 3) (2 3) (3 3) (4 3) (5 3) (6 3) (7 3) (8 3) 
                        (1 2 s p) (2 2 s p) (3 2 s p) (4 2 s p) (5 2 s p) (6 2 s p) (7 2 s p) (8 2 s p) 
                        (1 1 s r) (2 1 s n) (3 1 s b) (4 1 s q) (5 1 s k) (6 1 s b) (7 1 s n) (8 1 s r)))

!(lots_of_pawns_in_home_row)

\end{verbatim}


The following function \textbf{xy_box} function contrast is another illustration of the difference between using match with collapse versus ponderous old school tail recursion search. The first version inputs the piece color (gold for AI, silver for player) and rank, and outputs coordinates for the location of all pieces that match. The first version is really all you need in MeTTa. 

The second version displays the complexities of searching through lists with tail recursion.

\begin{verbatim}
; HIDE-RUN
;***********************************************************************
; Function:    xy_box
; Description: Find the coordinates of a piece on the board.
; Input:       A piece descriptor in the form: (color rank)
; Output:      Coordinates (x y) where the piece is located
;***********************************************************************
(= (xy_box ($PieceColor $PieceRank))
       (match &self (square $x $y $PieceColor $PieceRank) ($x $y))) 

;***********************************************************************
; obsolete comparison versions...
;
; (xy_box_recursive)
;     1. Searches the board for a square containing the specified piece.
;     2. Returns the x/y coordinates of the square if a match is found.
;
;***********************************************************************
(= (xy_box_recursive ($PieceColor $PieceRank) $Board)
  (let*
    (
    ; assign variables for the next piece
    ($next_square (car-atom $Board)  )
    ($next_x (nth 1 $next_square) )
    ($next_y (nth 2 $next_square) )
    )
    ; then examine next square
    (if (== (size-atom $next_square) 2) 
      ; empty square, skip
      (xy_box_recursive ($PieceColor $PieceRank) (cdr-atom $Board))
      ; else check this piece
      (let* (
            ($next_color (nth 3 $next_square) )
            ($next_rank  (nth 4 $next_square) )
            )
            (if
              (and (== $next_color $PieceColor)
                   (== $next_rank $PieceRank))
                ; return x and y of the found piece
                ($next_x $next_y)
                ; else keep checking the board
                (xy_box_recursive ($PieceColor $PieceRank) (cdr-atom $Board)))))))
\end{verbatim}


\begin{verbatim}
!(S) ;reset for demo

!(reset-pieces ((1 8 g r) (2 8 g n) (3 8 g b) (4 8 g q) (5 8 g k) (6 8 g b) (7 8 g n) (8 8 g r) 
                        (1 7 g p) (2 7 g p) (3 7 g p) (4 7 g p) (5 7 g p) (6 7 g p) (7 7 g p) (8 7 g p) 
                        (1 6) (2 6) (3 6) (4 6) (5 6) (6 6) (7 6) (8 6) 
                        (1 5) (2 5) (3 5) (4 5) (5 5) (6 5) (7 5) (8 5) 
                        (1 4) (2 4) (3 4) (4 4) (5 4) (6 4) (7 4) (8 4) 
                        (1 3) (2 3) (3 3) (4 3) (5 3) (6 3) (7 3) (8 3) 
                        (1 2 s p) (2 2 s p) (3 2 s p) (4 2 s p) (5 2 s p) (6 2 s p) (7 2 s p) (8 2 s p) 
                        (1 1 s r) (2 1 s n) (3 1 s b) (4 1 s q) (5 1 s k) (6 1 s b) (7 1 s n) (8 1 s r)))

; look for gold bishops using collapse and match
!(collapse (xy_box (g b)))

; look for gold bishops by search through supplied board; this version only finds 1 bishop.
!(collapse (xy_box_recursive (g b)  
      ((1 8 g r) (2 8 g n) (3 8 g b) (4 8 g q) (5 8 g k) (6 8 g b) (7 8 g n) (8 8 g r) 
                        (1 7 g p) (2 7 g p) (3 7 g p) (4 7 g p) (5 7 g p) (6 7 g p) (7 7 g p) (8 7 g p) 
                        (1 6) (2 6) (3 6) (4 6) (5 6) (6 6) (7 6) (8 6) 
                        (1 5) (2 5) (3 5) (4 5) (5 5) (6 5) (7 5) (8 5) 
                        (1 4) (2 4) (3 4) (4 4) (5 4) (6 4) (7 4) (8 4) 
                        (1 3) (2 3) (3 3) (4 3) (5 3) (6 3) (7 3) (8 3) 
                        (1 2 s p) (2 2 s p) (3 2 s p) (4 2 s p) (5 2 s p) (6 2 s p) (7 2 s p) (8 2 s p) 
                        (1 1 s r) (2 1 s n) (3 1 s b) (4 1 s q) (5 1 s k) (6 1 s b) (7 1 s n) (8 1 s r))
      ))

\end{verbatim}

\textbf{Chess Move Validation}

We now turn to a single function named "clear_route", the program's core chess piece move validator. We'll take a look at definitions for just the king and queen below.

\begin{verbatim}
; -HIDE-RUN

; Function:    clear_route
; Description: Validates whether a specific chess piece can legally move from a 
;              source square to a destination square, based on the rules of 
;              movement for each piece type. This function ensures that:
;              - The move adheres to each piece's geometry (e.g., bishop diagonals, rook lines).
;              - All intermediate squares (for sliding pieces) are unoccupied.
;              - Pawn-specific rules (direction, captures, initial two-step) are honored.
;
; Input:       Two expressions:
;              1. A piece located at a specific coordinate in the form: (x y color rank)
;              2. A destination square in the form: (x y)
;
; Output:      Boolean (True if the route is legal and clear; otherwise False)
;
; Notes:       
;   * Before calling this function, ensure that the source and destination are distinct,
;     and that if the destination is occupied, it contains an opponent’s piece.
;   * This function handles routing logic only. Occupancy checks must be performed externally.
;   * Supports: king, queen, rook, bishop, knight, pawn (both silver and gold variants).

\end{verbatim}

In MeTTa \textbf{we can define a single function as many times in as many variations as needed}. The simple strategy is to define one version of the "clear_route" function specifically for each piece type. A list composed of the starting square x-coordinate, y-coordinate, piece color, and piece type is the first argument. The second argument is the desired destination (x y) square, whether occupied or unoccupied. If occupied, prior to calling "clear_route" we must ensure the destination is occupied by the opponent. If the move is valid return True. If an invalid move, return False. 

Since the king can only move one square at a time in any direction, the "clear_route" definition for the king is relatively straightforward. A "k" is hard-coded in the first input parameter list along with variables for the x-coordinate and y-coordinate, so this version of the "clear_route" function will only be invoked if indeed the starting square is occupied by a king. 

The "clear_route" functions in the chess program are a good exercise in MeTTa's functional style boolean logic with the comparison operators appearing first. A code editor with colored and balanced parenthesis helps a lot to keep nesting straight.

\textbf{let*} firstly creates and assigns variables for the coordinates of the destination square. The second argument of \textbf{let*} in this case is an \textbf{if} construct, and if true (ie., the king is moving one square only any direction), the \textbf{if} construct -- and thus the \textbf{let*} -- returns the symbol True, else returns the symbol False.

\begin{verbatim}
; HIDE-RUN
; King: Moves one square in any direction.
(= (clear_route ($X1 $Y1 $Color k) $destination)
      (let*  (
      ($X2 (nth 1 $destination) )  
      ($Y2 (nth 2 $destination) )          
             )
      (if (and 
         (or      
          (== $X2 $X1)
          (or 
          (== $X2 (- $X1 1))
          (== $X2 (+ $X1 1))
          )) 
         (or
          (== $Y2 $Y1)
          (or 
          (== $Y2 (- $Y1 1))
          (== $Y2 (+ $Y1 1))
          )) 
          )
        True
        False)))

\end{verbatim}

\begin{verbatim}

; test if king can move up one square
!(clear_route (4 1 s k) (4 2))

; test if king can move diagonally (no)
!(clear_route (4 1 s k) (6 3))

\end{verbatim}

The "clear_route" function for the queen is defined in terms of being able to move like both the rook or bishop, as per the rules of chess.

Important to note: some of the "clear_route" functions use the above "square" atoms for speed, as described, so prior to calling "clear_route" the "reset-pieces" function must be called (it happens that the "clear_route" version for the king above does not use the "square" atoms).

\begin{verbatim}
; HIDE-RUN
; Queen:  Either rook or bishop move works.
(= (clear_route ($X1 $Y1 $Color q) $destination)
      (if (or (clear_route ($X1 $Y1 $Color b) $destination)
               (clear_route ($X1 $Y1 $Color r) $destination))
           True
           False))


\end{verbatim}

\begin{verbatim}

!(S) ;reset for demo

; reset the "square" small atoms with a board mock up so that queen can move vertically like a rook but is blocked for a bishop type diagonal move. The board mock up is the starting board with a pawn that moved from (4 2) to (4 4) allowing a vertical path for the queen.

!(reset-pieces ((1 8 g r) (2 8 g n) (3 8 g b) (4 8 g q) (5 8 g k) (6 8 g b) (7 8 g n) (8 8 g r) 
                        (1 7 g p) (2 7 g p) (3 7 g p) (4 7 g p) (5 7 g p) (6 7 g p) (7 7 g p) (8 7 g p) 
                        (1 6) (2 6) (3 6) (4 6) (5 6) (6 6) (7 6) (8 6) 
                        (1 5) (2 5) (3 5) (4 5) (5 5) (6 5) (7 5) (8 5) 
                        (1 4) (2 4) (3 4) (4 4 s p) (5 4) (6 4) (7 4) (8 4) 
                        (1 3) (2 3) (3 3) (4 3) (5 3) (6 3) (7 3) (8 3) 
                        (1 2 s p) (2 2 s p) (3 2 s p) (4 2) (5 2 s p) (6 2 s p) (7 2 s p) (8 2 s p) 
                        (1 1 s r) (2 1 s n) (3 1 s b) (4 1 s q) (5 1 s k) (6 1 s b) (7 1 s n) (8 1 s r)))

; test if queen can move vertically
!(clear_route (4 1 s q) (4 2))

; test if queen can move diagonally -- no since there is a pawn in (5 2) so no clear/valid path
!(clear_route (4 1 s q) (8 5))

\end{verbatim}



\subsubsection{Putting It All Together}

\begin{verbatim}
; ok
\end{verbatim}

Summary


\end{document}
